\documentclass{easychair}
\usepackage[utf8]{inputenc}
\usepackage{authblk}
\usepackage{cite}
\usepackage{authblk}
\usepackage{hyperref}
\usepackage{xcolor}
\usepackage{xspace}
\usepackage{mdframed}

\newcommand{\comment}[3]{\begin{mdframed}{{\textcolor{#1}{#2: #3}}}\end{mdframed}}
\newcommand{\yoni}[1]{\comment{red}{Yoni}{#1}}
\newcommand{\ahmed}[1]{\comment{red}{Ahmed}{#1}}
\newcommand{\makai}[1]{\comment{red}{Makai}{#1}}
\newcommand{\lazybvtoint}{\texttt{lazybv2int}\xspace}
\newcommand{\smtcomp}{SMT-COMP\xspace}
\newcommand{\smtlib}{SMT-LIB~2\xspace}
\newcommand{\qfbv}{QF\_BV\xspace}
\newcommand{\qfufnia}{QF\_UFNIA\xspace}
\newcommand{\msat}{mathsat\xspace}
\newcommand{\cvcfour}{CVC4\xspace}
\newcommand{\smtswitch}{smt-switch\xspace}

\begin{document}

\author{
		Yoni Zohar\inst{1}\and
		Ahmed Irfan\inst{1} \and
		Makai Mann\inst{1} \and
		Andres N\"otzli\inst{1} \and
		Andrew Reynolds\inst{2}\and
		Clark Barrett\inst{1}
}
\institute{
  Stanford University \and
  The University of Iowa
 }

\title{Lazybv2int at the SMT Competition 2020}

\maketitle


\noindent
\begin{abstract}
\lazybvtoint is a new prototype SMT-solver, that  will participate in the incremental and non-incremental tracks of the \qfbv logic.
\end{abstract}

\paragraph{Overview}
\lazybvtoint is a prototype SMT-solver for the theory of fixed-width bit-vectors and uninterpreted functions.
This is the first year it will compete in \smtcomp.
It will participate in the incremental and non-incremental \qfbv tracks.

%basic idea
The basic algorithm of the tool relies on a translation from bit-vectors to
non-linear integer arithmetic with uninterpreted functions, followed by
a CEGAR loop \cite{cegar} that lazily instantiates bit-vector axioms over the translation.
The idea of using integer reasoning for bit-vector solving is not new (see,
e.g., \cite{DBLP:journals/entcs/BozzanoBCFHKPS06,DBLP:conf/fmcad/BackemanRZ18}), however, it is worth revisiting due to recent improvements in solvers for non-linear integer arithmetic.

We expect this solver to perform better on benchmarks that involve arithmetic
bit-vector constraints and large bit-widths because the encoding of arithmetic
constraints is straightforwad and independent of bit-width, as opposed to the
encoding of bitwise constraints which is less natural and hindered by larger
bit-widths.
% Makai: minor re-wording, feel free to edit above or switch it back to the text below
% because these seem
% more suitable for the approach of this tool.

The tool is open-source and is available at \url{https://github.com/yoni206/lazybv2int}.

\paragraph{Dependencies on Other Tools}
%
To parse the input problem, \lazybvtoint employs \msat's
parser through an API \cite{mathsat5}.
To solve the translated arithmetic problem it uses \cvcfour \cite{cvc4}.
In some cases, \cvcfour is called on an extension of the original bit-vector problem.
%
The interface to both external solvers uses
\smtswitch \cite{smtswitchgithub}.

According to the rules published by the organizers
of \smtcomp 2020, \lazybvtoint is a {\em wrapper tool} (see \cite{rules20})
and not a {\em portfolio solver},
and thus is allowed to compete.
This was confirmed in a private communication with the competition organizers.

%technical details
\paragraph{Technical Details}
\lazybvtoint works as follows.
The input \qfbv formula $\varphi$ is translated into a
\qfufnia formula $\varphi'$.
$\varphi'$ is obtained from $\varphi$ by eliminating
bit-vector operators.
For arithmetical operators, this is standard.
The bit-wise operators other than bit-wise conjunction, left
and right (logical) shift, and negation are first
eliminated in a preprocessing stage using other bit-vector operators.
Bit-wise negation has a standard arithmetical interpretation which is utilized.
Bit-wise conjunction as well as shift operations are replaced by uninterpreted functions.

The translated formula is solved using a CEGAR-loop that refines the current formula by adding lemmas.
The procedure is complete because the lemma schemes that are used are complete.
%
These include some basic properties of the abstracted operations
(e.g., idempotence of bit-wise \emph{and}), and in the worst case,
they include a full expansion of the operation (using
$ite$ operations and summations).

In each step of the loop, there is one additional check performed by \cvcfour's \qfbv solver:
\yoni{ahmed please write here everything about sat-checker and sat-checker-filter, and also about unsat cores.}


%conclusion
\paragraph{Conclusion and Acknowledgments}
To conclude, this is a prototype experimental tool that is aimed
to serve as a playground for arithmetic-based techniques
for bit-vector solving.
Incorporating such techniques in a full-fledged solver is left for future work,
and is planned for when these techniques are better
understood and evaluated using this tool.

%acknowledgments
We would like to thank the \cvcfour and \msat teams for allowing us
to use their tools without any issues.
In particular, we thank Alberto Griggio for clarifying
the relevant aspects of the \msat license,
and to Aina Niemetz and Mathias Preiner for helpful tips
regarding benchmarking and evaluating bit-vector formulas.
We also thank the competition organizers,
Haniel Barbosa, Jochen Hoenicke, and
Antti Hyvarinen for clarifying the status of this tool for the competition and willingly accepting it as a new participant.



  \bibliography{smtcomp-2020}
  \bibliographystyle{plain}

\end{document}
